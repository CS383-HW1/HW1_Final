\documentclass[12pt]{report}
\usepackage{geometry}
\usepackage{titling}

\newcommand{\subtitle}[1]{
	\posttitle{
		\par\end{center}
		\begin{center}\large#1\end{center}
		\vskip0.5em}
}
\setcounter{tocdepth}{2}
\title{Homework 1}
\subtitle{CS 383 - Group 3}
\author{
Mason Fabel \\
\and Morgan Holbart \\
\and Ronald Rodriguez \\
\and Tessa Saul \\
\and Lance Wells \\
\and Zachary Yama \\
}
\date{\today}
\geometry{margin=1in}

\begin{document}

\maketitle

\tableofcontents
Document Typsetting and Editting: \textbf{Mason Fabel}
\clearpage

\chapter{Application Domain Study}

\begin{section}{Related Applications}

\begin{subsection}{Nethack - Mason Fabel}
For the application domain study, I chose to study the classic roguelike
game Nethack. I chose this title for a number of reasons. First of all,
the roguelike genre is, if not the direct predecesor, the inspiration for
the majority of tile-based dungeon crawling games, among others. This
being so, I chose Nethack as it a mature and heavily-played roguelike.

For my study, I am going to play a session of Nethack and record a
condensed version of my adventures. I will then discuss the elements that
I think can be used successfully in our project, followed by the elements
that I think can not.

\begin{subsubsection}{Gameplay}
Nethack opens with a pretty straightforward selection of menus asking the
player for their class, race, and gender. I selected a human male Tourist,
as I am merely making a small tour of this game for this study. A short,
three paragraph intro is dropped on me, telling me to descend the
dungeon and fetch some amulet.

And that's where the handholding stops. I'm dropped into the game with the
oh-so-helpful message "Be careful!", and nothing else. Some quick fiddling
will reveal that the various keypresses correspond to different actions,
and that pressing '?' will bring up some very cryptic help. Fortunately,
this is far from my first time in Nethack, so I'm over the gigantic
learning curve needed to even make sense of the screen and perform basic
actions.

Thus, off I go, looking for the amulet. The game itself was pretty simple
to watch. Explore the map, look for the stairs to go deeper, run away from
monsters until my pet kills them and I loot the corpses. In this manner I
cleared a few levels and then called it a day.
\end{subsubsection}

\begin{subsubsection}{The Good}
There are a lot of good, fun elements to Nethack, even if I didn't come
across too many of them in the short period I had to play and study it.
Some of these features are a good match for the project we are making,
while others are not so good.

The first element of note which could be put to good use in our project
is the fact that each dungeon level is randomly generated, and so are
the items, to a certain extent. What Nethack does is layout a general
level generation scheme (rectangular rooms connected by corridors during
the early levels), and then randomly create floors which follow those
rules. Thus, the game isn't at the complete mercy of the random generator,
but can set and follow certain expectations while still providing variety
and a unique experience each game.

Furthermore, the items are partially randomized. While each game of
Nethack will have the same potion of acid, ring of levitation, or other
items, each game the effect of, say, the red potion is different.
Play once, the red potion is healing. Play again, it's hallucination, or
perhaps grease. While this particular mechanic hinges on the fact that a
major part of a game of Nethack is discovering what your items are and
what they do, the general principle of using controlled randomization to
add variety to different playthroughs of the game is praiseworthy, and
should, in my opinion, be used throughout our final project.
\end{subsubsection}

\begin{subsubsection}{The Bad}
While Nethack is a fantastic game, some of its features would translate
extremely poorly over to our project. After all, Nethack is a single-player
dungeon crawler with an emphasis on mechanical depth, and our project
is a cooperative multiplayer game, which calls for what can be very
different features.

The most obvious feature to discard would the be the interface. While
Nethack functions quite well once you grow accustomed to it, the controls
are very obtuse and extremely difficult to learn. To further complicate
the issue, most of the information regarding what is happening and all of
the information regarding what actions the player can take are completely
hidden, unless one already knows a complicated maze of menus and actions.
When we build our project, we should make it very clear what is currently
happening and what actions the player can respond with.

Another feature of Nethack that should definitely be dropped from our
project is permanent player death. In Nethack, if you die then your game is
over, permanently, and you must start again. While that contributes in a
large way to the charm of the game, this is completely out of place in a
multiplayer game, where a single mistake could force the player out of an
otherwise pleasant game. Furthermore, when playing with other people it
is very often fun to compare progress and try to one-up your friends.
Having the game threaten to wipe that out forever is typically not a good
feature for such games.
\end{subsubsection}

\end{subsection}

\begin{subsection}{Crawl - Morgan Holbart}
I chose to conduct my domain study on the recently released beta version
of the game ``Crawl''. It has some roguelike elements, but more closely
resembles a dungeon crawler. The game is a multiplayer dungeon
crawler where 4 players compete to reach level 10 and defeat the boss
first.

\begin{subsubsection}{Gameplay}
The game has, as expected for its genre, very simple controls and
graphics. The value of the game comes from its unique competitive nature
and ever
changing strategy. The twist to this dungeon crawler is that one player
is a human, and the other 3 players play the monsters in the dungeon. To
start each player chooses a god to worship (which gives a passive bonus),
and the game begins. One player is randomly selected and sent off to
explore the dungeon.

Once the player moves from room to room, the other players are given the
ability to control monsters in the room and begin attacking the human
player. They are rewarded blood points for damage dealt to the player, and
the player is granted experience for killing monsters. The goal of the
human is to try to get as much experience as possible to reach level 10
and try to beat the boss, with careful consideration of the fact that for
each level he gets, when he moves on to the next dungeon floor, every
other player will get corresponding points to level.
Once the human is killed, whoever dealt the killing blow
will take their turn as a human, starting at level 1 (or the previous level he
achieved) and can begin moving on.

All the blood dealt as a monster can be turned to money as a human, and
used to buy gear if you find the store that is on each level. The game
balances itself out by making your monster stronger for every level
another player gains, and by only being able to buy gear by earning money
while playing as a monster.
\end{subsubsection}

\begin{subsubsection}{Good Ideas}
The game is fast paced, there is limited setup and vast replayability. It
is focused more on the competitive nature and goal of winning than it is
the roguelike or RPG elements. If every player knows what they are doing,
the games can end anywhere between 15 and 30 minutes with plenty of
push-pull for first place.

The game balances itself, for if you are a human for a long duration the
monsters get enhanced, and if you are the monster for a long duration you
will be enhanced until you surpass the other monsters and can more easily
become the human.

The game is procedurally generated and unpredictable, you never know what
the room you are about to enter will have, it can have an assortment of
traps (controllable by monsters also), monsters, and treasure.

The game has a fairly strong AI, that in its hardest difficulty is fairly
difficult to beat, and plays the game pretty similarly to a human, give or
take a couple inconsistencies in the behavior.
\end{subsubsection}

\begin{subsubsection}{Bad Ideas}
The game has some balance issues most likely because of the beta state of
the game. Some monsters are fairly  overpowered compared to others, and
some of the gear can be overpowered, as well. Though careful selection of
monster ugprades can help to counter certain weapons, you can only upgrade
if you have points and at the end of each floor, so the unbalance can
carry the human until he moves on to the next floor.

The boss can be unbalanced based on the weapons and abilities available to
the humans (which are randomly generated). A ranged weapon trivializes the
boss, and a melee weapon is basically impossible to win with against any
competent players because of how the boss is designed.

Level 1 monsters are too weak, and whoever gets human first basically gets
free reign over the first dungeon floor until people can upgrade the
monsters.
\end{subsubsection}

\begin{subsubsection}{What Will Translate Well}
The game is a floor based dungeon crawler that is both cooperative and
competitive. Players are working together as monsters, but against the
human, but because of the turn based nature of having control of the human,
it can be difficult to form alliances, thus retaining the competitive
balance. 

The game did procedural generation very well, though sometimes it can be
unbalanced if the current human lands health potions multiple rooms in a
row and another does not.

The game ends fairly quickly, which may or may not be good depending on how
in depth we want the RPG elements to be. If we plan on a fairly low depth
RPG system, this game has replayability and speedy gameplay down.

You gain experience based on the length of the game, which
unlocks new items, gods, and monsters for new playthroughs of the  game.
This adds to the
replayability, allowing you to add strategic depth over time, keeping the
game simple to start.
\end{subsubsection}
\end{subsection}

\begin{subsection}{Teleglitch - Ronald Rodriguez}
For my application domain study, I chose to study Teleglitch. Teleglitch 
is a rogue-like top-down shooter/slasher action game. I chose this game 
as I felt it was somewhat similar to our goal of a tile-based rpg without 
``waiting'' battles.

I will complete the first few levels of Teleglitch and record my thoughts 
about which features I feel would and would not work for our 
game. 

\begin{subsubsection}{Gameplay}
From the start menu of Teleglitch, the player can choose to play, read a 
tutorial, or choose a level they have previously completed. Once the 
user selects play/continue, they are dropped into the world. Players can
move  the character around the screen using standard keyboard/mouse
controls. 

The way levels are designed and generated in Teleglitch is interesting. Not
only are levels randomly generated each time the user plays, but there is a
neat mechanic where most of the level remains hidden until your character 
actually sees that section by navigating over to it. This causes great 
moments of surprise when you travel to a new part of the level only to be 
met by a horde of enemies. 

The core of the gameplay comes in the form of killing enemies using various
weapons. The character begins the game with a sword that he can use to
melee  enemies with. Other weapons like guns and bombs are obtained via
loot drops from downed enemies and item boxes. One of the main mechanics of
the game comes in the form of being able to combine certain loot items into
more useful weapons. For instance, you may find some nails in an item box
at one point, and then a bit later find a baseball bat after you kill an
enemy. The game allows you to simply combine these two items into a
nail-bat which gives it a damage boost over a standard baseball bat. There
are multitudes of these potential item combinations. 

The game also incorporates elements of the survival horror genre in the
form of ammo being somewhat of a rarity, encouraging the player to think
very carefully about when they choose to use their guns insted of their
melee weapons.
\end{subsubsection}

\begin{subsubsection}{The Good}
The shooting mechanics in Teleglitch are exceptional. When a gun is fired
at an enemy, a noticable ``blur'' effect happens to the character and the
bullet itself. This effect adds a palpable ``weight'' to the shots, making
them visceral, impactful, and highly satisfying. 

The loot combining element is an excellent mechanic. It causes players to
think critically about the potential ways they can combine the multitude of
items in their inventory. A great sense of satisfaction comes when you
think of a way to combine two things and the game allows you to do so
exactly as you envisioned. 

The level designs are very cool, having a very industrial, apocalyptic
aesthetic. The levels are somewhat minimal in design, which lends well to
the game overall, as there is rarely any unnecessary clutter obstructing
you from attacking your foes or getting around the environment. 

While it has been a complaint of many industry reviewers, I think the
low-fi look of the game overall is quite charming. A game like this doesn't
need to have super high-res graphics. I felt it would add almost nothing to
the game at all. 
\end{subsubsection}

\begin{subsubsection}{The Bad}
The application of ``ammo rarity'' in this game only ended up frustrating
me, rather than giving me a feeling of tension. With Teleglitch, where the
shooting is easily the highlight of the entire game, implementing a
mechanic that lessens the amount of times you will use that awesome gunplay
seems silly to me. 

The melee combat in Teleglitch is nowhere near as satisfying as the
gunplay. There is a noticable lack of ``feedback'' when the melee weapons
connect with enemies, sometimes leaving you unsure if you actually hit the
enemy at all, since most take more than one hit to kill.  

In Teleglitch, when you want to see the whole map of the level you are
currently on, you need to press a button that actually zooms out from the
current room you're in and shows you the whole map. I felt this feature
broke the immersion of the game. I think an onscreen mini-map that 
could be toggled on and off would have been a much better choice both for
Teleglitch and for our game.
\end{subsubsection}

\end{subsection}

\begin{subsection}{Wanderlust: Rebirth - Tessa Saul}
I chose to look at the tile-based RPG Wanderlust: Rebirth. I chose this
game because of the games I already had, this one best fit the discription
of what we are trying to do. This game has a single-player mode and a co-op
mode with up to four people. 

I played through the tutorial and first couple levels in single-player
mode. 

\begin{subsubsection}{The Good}
This game has a simple and quick class selection system. There are four
classes and two genders. Each combination has a detailed portrait. Besides
that, the only thing you have to pick to make a new character is a name. 
This makes starting a new co-op game very fast.  
Leveling up was also very easy and fast. The game gives you points to
distribute among your skills. Depending on how you place them, you would 
have very different builds. The game has a `lobby' where you can
reassign all of the points that you have earned into different builds, and
then try them out. Another thing I liked about the game is that when I died
multiple times, my character was stunned for a few seconds, and then I got
to continue playing. I didn't even realize that I died until the end of
the level when it deducted points for the death. This means that even if
you are very bad at the game, it will not interrupt the fun you are having
to point out how bad you are. The game fills in your abilities by giving
you AI companions as well. This is nice because you do not have to play a
class to have its abilities, like healing. 
\end{subsubsection}

\begin{subsubsection}{The Bad}
The inventory and crafting systems in this game feel very clunky. The
tutorial had me create an item from a recipe, but I had to select all the
items for the recipe by hand in my inventory. This was frustrating because
the items only had little icon type images to represent them, so I had to
look at each item to see if it was one I needed for the recipe. If we have
an inventory system, I think we should look at other games for ideas. 

The controls on the game feel `spammy'. By this, I mean that you can pretty
much press random buttons rapidly and get about the same result as if you
planned out an attack. This is emphasized by the fact that if you play
single-player, the game gives you three AI companions that are the other
classes. They swarm around you when you hold still, and attack everything
at the same time as you, so it is hard to see what you are doing. 
\end{subsubsection}
\end{subsection}

\begin{subsection}{Binding of Isaac - Lance Wells}

For my section regarding the application domain study, I have decided to
review general concepts regarding our game's genre, in addition to drawing
from specific examples introduced in popular ``Roguelights''.

To begin, our general game concept runs closely in the vein of a
traditional ``Roguelike'', that is, our game borrows several features from
the genre introduced by the game Rogue. Included in our description
specifically are:

\begin{itemize}
\item Tile-Based Aesthetic and Generation
\item Dungeon Crawling Gameplay
\item Role-Playing Game thematics
\end{itemize}

To facilitate the description of these items, the following is a list of
interpretations of these features from the Roguelight ``Binding of Isaac''.

\begin{itemize}
\item \emph{Tile-Based Aesthetic and Generation}

Binding of Isaac (or more recently, the Rebirth edition), features a
tile-based dungeon layout. The game limits the range of tile-values to only
a select few (Passable Terrain, Impassible Terrain, Destructible Terrain,
Spikes), while altering tilesets throughout each themed floor set. This
strategy effectively simplifies the gameplay while still allowing for a
large variety of room layouts.

\item	\emph{Dungeon Crawling Gameplay}

Rather than adopting a standard free-movement-by-floor interpretation of a
dungeon crawler. Binding of Isaac chooses to instead only display one
``room'' at a time to the player. This effectively lowers the complexity in
displaying an entire ``floor'' simultaneously while still maintaining a
maze-like floor-by-floor dungeon crawler.

\item	\emph{Role-Playing Game thematics}

Binding of Isaac chooses to abandon conventional leveling systems while
instead functioning on a skill-by-skill basis. Each character is influenced
by a series of simple numeric-based "skills" that change the gameplay
slightly (e.g. movement speed, projectile speed, luck, etc.).
Simultaneously, a player may alter their experience by picking up various
stat-altering or otherwise gameplay-altering items ranging from mild to
extreme in effect. What more, Binding of Isaac maintains a small portion of
``risky'' decisions often incorporated into roguelikes similar to
random-effect pills, varying-effect  cards, and health-depleting ``Deals
with the Devil''.
\end{itemize}
\end{subsection}

\begin{subsection}{Quest of Dungeons - Zachary Yama}
For my application domain study I chose to analyze the dungeon crawler,
Quest of Dungeons. I chose this title because it’s a more recent, indie
approach to the genre of dungeon crawlers and we will be approaching our
game from the same sort of perspective. It also happens to be fairly
popular, so it would be interesting to see what features make this game a
successful dungeon crawler.
 
The basis of this study will be drawn from a single 15 minute campaign
session. Throughout the session I will record any important game features
and interesting experiences I find. Once a reasonable list has been
compiled I will break it into three sections: how the gameplay directs 
the player, what elements could be included in our game, and elements that
probably shouldn’t be included in our game.
 
\begin{subsubsection}{Gameplay and Direction}
On game launch, the player is presented with a generic and straightforward
navigation menu. Once a campaign is started, the player is asked to pick a
class. There are four choices: wizard, warrior, assassin, and shaman. After
a class is selected, a short humorous cut scene is presented where the
other three classes that were not selected suggest you take on the world’s
most evil villain and plunge into his dungeon alone. Reluctantly the class
that was chosen by the player exits the scene and the actual game begins.
In my case, it was the Wizard.
 
After being dropped into the game, I noticed a generic set of GUI elements:
health, mana, a minimap, a place to set skills for quick use, and a set of
buttons to open the inventory, stats, and quest menus. After playing with
the keyboard for a bit all the movement controls become clear. The game
uses WASD movement and a point and click system to perform actions.
 
Now that I had an idea of how the mechanics worked I decided to go see if I
could kill something. In fact I could; with giant fireballs. I also noticed
there were many breakable objects lying around. Many yielded gold and even
items and spells I could equip. After running around for a while I found a
large stone that began a quest to kill so and so with some neatly included 
directions. After spending about five minutes heading off on an interesting
adventure I decided to be done and quit the game.
\end{subsubsection}
 
\begin{subsubsection}{Useful Features}
Many of the features I experienced in Quest of Dungeons were really fun and
interesting. Though, with such a short session my first impressions may not
do all elements justice. Regardless, there are some elements that would
blend well in our game, and others not so much.
 
The most interesting thing I came across was the spell system. There aren’t
any skills, you have to find them from drops. What’s particularly
interesting is that you can use spells that seem more like something a
warrior would use. The catch is that the spell does damage scaled off of a
specific stat. For example, the fireball skill I had scaled off of
intellect by a rating of B, where the rest of the stats it scaled off of
were F.
 
Another game element I found inspirational was the fast paced combat
system. Monsters do only move when you do, but the way the game behaved was
very polished and quick. This is somewhat attributed to the WASD movement
and the snappy response of having an active spell that is used only by
pointing and clicking. I didn’t need to select anything or think about what
to do; I just did it.

Lastly, the questing system was somewhat interesting. Being able to go out
adventuring and randomly find a stone that gives you newfound direction was
rather refreshing. It would be nice to include some sort of questing system
in our game so that the player isn’t suck with just running around the
dungeon waiting for the last level to come.  
\end{subsubsection}
 
\begin{subsubsection}{Harmful Features}
This game was extremely fun to play, but my experiences were with a single
player campaign. There are some components that clearly wouldn’t work for
our multiplayer version, which is what we intend to do.
 
One of these features is locking a player in a room. For a multiplayer
system this would be more harmful than beneficial. If a party accidently
overextends into a higher level room, they would likely die and 
lose valuable experience and have to restart at no real fault of their own.
There would also be inconvenient communication requirements to make sure
that the whole party is ready to move on, and which room they decide as a
whole to move to next. These two elements will needlessly slow down the
players.
 
Another feature that shouldn’t be included is permanent death. If a player
dies, they are required to start completely over. This would prevent
players from wanting to play with each other simply because they don’t
know how prepared the other player really is. This again slows party
progress and requires additional unnecessary communication.
\end{subsubsection}
\end{subsection}
\end{section}

\chapter{Project Design and Direction}

\begin{section}{Individual Brainstorming}

\begin{subsection}{Mason Fabel}
Our project is to be ``a turn-based co-operative multiplayer
'dungeon-crawl' role-playing game'', combining the ideas of an adventurous
science fiction setting and the premise of a lowly office worker rising
through the ranks to become the master of his own destiny.

While contemplating these goals, I came up with a few central concepts and
observations I would like to see incorporated into the final project.

\begin{subsubsection}{Tone and Theme}
While there are a lot of directions the overall tone and feel of this
projecan take, I believe the best direction would be a lighthearted
combination of comedy, in the vein of Douglas Adam's Hitchhiker's Guide to
the Galaxy, and action/adventure, such as in Star Wars IV.

Taking this approach instead of a realistic approach gives the team more
freedom as to the shape of the final project. As we are not constrained by
accurately simulating reality, we can instead focus on creating a game that
is fun to play and avoid complicating areas that would not benifit from
such attentions.

Keeping a light tone will allow us to avoid dark or overly thoughtful
material. While these things have a time and a place, the purpose
of this project is to learn how to organize and create large pieces of
software, not to create an artistic or philosophical statement. By avoiding
such material, the focus is kept on the engineering, where it should be,
instead of the content, which is a matter for a different class.
\end{subsubsection}

\begin{subsubsection}{Content Generation}
The main purpose of this class is to learn software engineering, or how
to organize and build large software projects. While we have chosen to do
this in the form of a game, the purpose of this class is not game design,
but rather building a large application. Thus, whenever possible this
project should be moved away from creating game content and towards writing
code and organizing the development process.

To this end, I believe that as much game content should be generated on the
fly by the application wherever possible. This obviously covers simple
things, such as level layout, but it is not a hug jump of the imagination
to take this same aproach to many, if not all, of the elements that will
appear in the final product. Such examples may include NPCs, quest lines, or
equipment, but wherever we wish to have more than a handful of types this
sort of system is theoretically possible.

In addition to taking the burden of content generation away from the group
and giving it to the computer, this will have the further advantage of
allowing the size of the project to be scaled up (or down) by creating
different content generation methods for different sections of the final
game.

Finally, generating content will allow the final project, which is going to
be created by a small team in a short period of time, to have a lot of
content relative to programmer effort. Some finite amount of effort will
allow the final product to have a much larger amount of content, as
opposed to the more linear relationship involved when humans create
content by hand.
\end{subsubsection}

\begin{subsubsection}{NPC Social Networking}
Traditionally, most RPGs have used a system based upon levels and
experience points. While this works well in a lot of settings, it seems to
me that this transfers poorly over to the concept of the main character
being an office worker.

Thus, I would like to make progression in our final project not dependant
upone collect experience or gaining levels, but rather on a ``social
network'' of NPCs, for lack of a better term.

The basic idea is pretty simple. You have some ``relationship value''
with each NPC character. This value can be some positive or negative value,
with positive values meaning they like the player and a negative value
meaning they do not.

In addition to relating to the player, NPCs would relate to each other by
some percentage value between $-100\%$ and $100\%$. This would then serve
to modify the player's relationship with that NPC.

For example, the player has a $+50$ relationship with NPC A and a $-15$
relationship with NPC B. NPC A has a $10\%$ relationship with NPC B, and
NPC B has a $50\%$ relationship with NPC A. Thus, the player's final
relationship value with NPC A is $50+0.1(-15)=48.5$, and the player's
final relationship value with NPC B is $-15+0.5(50)=10$.

As the number of NPCs increases, a increasingly complex and interesting web
of characters can be created.

Now suppose that NPCs were tied to plot events or player progression.
Specifically, in order to advance further in the game, the player needs
to be promoted within the office or organization they work for, and this
can only be accomplished by gathering favor with their coworkers and
managers.

Furthermore, NPCs could be made to ignore the player until they reached
some relationship threshhold, thus causing the player to build relationship
with that NPS's ``friends'' in order to interact with them.

Finally, some method of gaining relationship with NPCs would need to be
implemented, whether a traditional ``kill 30 grues'' variety of quest or
some other, more original, method.

The result of this would be a varied and dynamic system (especially if this
network of NPCs was also generated by the game instead of being hard coded)
where the player is required to gain social goodwill until they finally
come to the attention of the top entities, thus recieving the final
promotion to CEO, or the final boss fight, or whatever the end game turns
out to be.
\end{subsubsection}
\end{subsection}

\begin{subsection}{Morgan Holbart}
\begin{subsubsection}{Setting}
As a cooperative or, at least, multiplayer game, the story and overlying
setting of the game is not very important, I think the best blend of the
predetermined sci-fi and bureaucracy themes would simply place the player
in a world that combine the two with no overt explanation. For example,
the player is an office worker in an alien immigration building.
This gives a setting, a floor based ``dungeon'' (the building), and easily
changeable enemies. The enemies can be any assortment of the aliens that
come through the place, which means they can look like or do anything we
decide to fit the game, instead of trying to stick to a certain lore.
\end{subsubsection}

\begin{subsubsection}{Game Design}
The game will pit a player and his office colleagues into the immigration
office with the desire to exterminate all of the aliens currently inhabiting
the immigration office. You will work your way up starting from floor 1 to
the top floor where the main boss will be. Each floor will have a multitude
of mini bosses, enemies, loot rooms, and puzzles. Because of the procedural
generation, each game will have an entirely different flow. The player and
his colleagues should have to work together to solve each floor, finding and
killing the miniboss to unlock the next level, and completing any puzzles
to collect the superior loot to assist with the next floors.

Because the game is multiplayer, permadeath is a difficult feature to
implement, no one wants to wait for the game to end, so instead every time a
player dies, when the next floor is reached by his allies, he will be
revived with a reduced maximum health, if someone dies 3 or some number of
times, the next death will be permanent. Any ally can however revive a dead
player and sacrifice his maximum health to do so.

The game will feature a variety of weapons, the basic weapons will be the
human weapons, and will be the most common, you can find alien weaponry off
the corpses of your fallen foes which is the second tier of loot, and the
third tier of loot will drop from puzzles and mini bosses. All loot will be
tradeable between party members.

The progression system will be based purely on gear and floor number, there
is no need for an XP system, the enemies will scale with their floor, as
will the quality of loot drops.
\end{subsubsection}

\begin{subsubsection}{Game Technology}
The game should have several mechanics that function to enhance the game and
demonstrate our coding prowess. As a dungeon crawler, procedural generation
is a must. All levels of the game should be procedurally generated, as
should all items, possibly monster abilities/AI.

The game should adopt multiplayer systems for both LAN and internet based
gameplay, it can implement lobby based matchmaking, or direct IP connection,
but should not require an authoritative server for controlling game logic. 
This will leave the game open to hacking and potential cheating, but without
any leaderboards and the desire for LAN support, this should be a non issue.
Each client can most likely handle everything on its own, with one specific
host that will manage the communication between the two.

The game needs to have a sophisticated enough AI to support both
singleplayer play (bot controlled allies) and enemies with a alterable
difficulty to keep the challenge for players of all skill level. The AI
logic will have to be controlled by the host client.
\end{subsubsection}
\end{subsection}

\begin{subsection}{Ronnie Rodriguez}
\begin{subsubsection}{Tone and Theme}
As my collegue has stated, a solid mix of humor and sci-fi action would be
an excellent direction to take the game in, especially since we have been
asked to keep the content ``G and PG'' rated. 

This begs the question, though, as to what brand of humor we should be
trying to use. In many comedic workplace-set films and TV shows, like The
Office, Office Space, Parks and Recreation, IT Crowd, etc., a palpable type
of sarcastic, droll, and deadpan humor is used. It's almost as if the
characters in these shows have just had all the life sucked out of them by
their boring, unfulfilling work, and almost everything that exits their
mouths is a bitter, sarcastic quip at something or someone. I think this
type of humor could translate well into a corporate-office set game. I
think it is entirely possible to do this, while keeping it PG rated. 
\end{subsubsection}

\begin{subsubsection}{Health Bar}
In many hack-n-slash RPG games, increasing in level gives you a bonus to 
your health bar. I think it would be cool to implement such a feature in
our game, but with a comedic twist. 

Assuming we adopt the deadpan style of humor popular in workplace-set
comedies populated by characters who, for all intents and purposes, seem to
have all but lost their will to go on, I think that instead of a ``health''
bar, it would be funny to have ``will to go on'' bar. 

For instance, instead of saying something like ``Health has been increased
by 5'' after going up a level, we could have something like ``Your will to
go on has slightly increased. So you got that going for you, which is
nice\ldots'' after our character receives a promotion within the company.
\end{subsubsection}
\end{subsection}

\begin{subsection}{Tessa Saul}
I would like to see a game where you can rebuild your character to test
choices. A lobby area where the player can arrearage the configuration of
their modifications is a good match for that. In the lobby, the character
cannot die, or cannot take damage. The lobby also has `training dummy' type 
enemies that will persist through or reappear after player attacks. The
lobby would allow the player to test out new parts in a safe area so they
can maximize their effectiveness.

In our group meetings, we talked about having the multi-player section of
the game involve dungeon areas. This means that you do storyline advancement
on your own, and then get your friends together to go fight off the
bad-guys. We talked about the idea of going to cyberspace to fight off
viruses, which would be the dungeon area. I like this idea because it means
that no-one actually dies, or gets hurt which is in line with the G or PG 
rating requirement. This idea is similar to a part of the game Shadowrun,
where you also visit cyberspace to get secret information and fight viruses.
In Shadowrun, you get to load different helper programs, hardware, and tools
to help you before you dive in. I think it would be fun to implement
different equipment pieces like these. 

If we do a cyberspace centric game, I think it would be fun to gain skills
in different programming languages, and things we have learned about in
class. That way your character can make their own viruses to match up
against the bad ones, or implement better coding techniques to thwart the
bad-guys. I think it would be easier to come up with jokes about this topic
because we all already know about it. Overall, this idea is flexible, so I
hope we can put a part of it in the game.
\end{subsection}

\begin{subsection}{Lance Wells}
The prompt in question asks that we merge two concepts:
\begin{itemize}
\item \emph{A Star-Warsish Sci-Fi in which you overthrow the evil empire}
\item \emph{A lowly office worker fending off paperwork and bureaucracy to challenge the CEO}
\end{itemize}

Blending the two ideas brings about a few questions:
Evil empire and a corporation to challenge? Why not work for the "bad guys"?
Paperwork to fend off and a Star-Warsish Sci-Fi setting? Why not fend off 
sentient paperwork?

I also have a general storyline-concept that I would like to propose:

\begin{center}
\emph{It's the year 272820 and the remnants of the human race have been
enslaved by the Conquesting and Expanding Organization, otherwise known as
the CEO.}
\end{center}

The only clear goal of the CEO is to mitigate the human spirit, and
assimilate the entirety of humanity into its dull corporation through the
most horrifying weapon imaginable: paperwork.

Maintaining a ``Will to Live'' (as per Ron's idea) is essential to surviving
the work-ridden future. Only by fighting the MAN (Malevolent Autonomous
Navigator), a super-intelligence deigned to direct the future of the Earth
towards a grim and monotonous future can the player(s) rescue the Earth and
avoid becoming soulless engines to a galactic war machine.
\end{subsection}

\begin{subsection}{Zachary Yama}
\begin{subsubsection}{Story and Theme}
I very much like the story written by our first two game idea entires. Much
moreso than my own, but either way, here's my take on it.

The basic premise of our game must include both office worker themes and
sci-fi themes. The morphed combinational theme that comes to mind is of a
scientist who develops weapons of mass destruction for a company you, the
office worker, have deemed unethical and evil. The corruption of the company
you work for is so expansive that you dare not to leave out of fear.
However, because you aren’t the only scientist, you find yourself caught up
in the evil mad scientists’ scene when you really don’t want to be. The only
co-worker you’ve been friends with and ever enjoyed working with has
accidently unleashed an aggressive stimuli into the complex because of
reasons unknown. His dying act of kindness was to give you the only
vaccination. Your quest is to make it to ground level and out safely,
find out what
happened to your friend’s bio-aggression project, and wipe out all infected
individuals. All along the way you will be engineering and developing new
weapons and traps, while dispatching strategies to take out mobs, solve puzzles,
and defeat bosses.
\end{subsubsection}

\begin{subsubsection}{Building Weapons}
An additional feature that I would like to propose is the way weapons are
created and generated within the game. A good equipment system that feels
alive is something that can really bring a lot to the table. Though random
generation is an option, allowing the player to build something of their
very own is equally if not more powerful. It creates additional immersion
and charm to the game, and invites the player to continue somewhat solely for
the sake of finding out what new cool weapon they can build by defeating
the next enemy.

For weapons and items, having the ability to construct your own from parts
you've found throughout your adventures is extremely interesting. I've
seen this done before in the game Loadout, which is a free third person
shooter. It employs about 6 total parts for any weapon, with about
10 varaiations each. These parts will modify the behavior of the weapon in
some way. For example, if you decided to use a scope you can zoom. If you
decide instead to use a laser scope, now you've got a laser to aim with. The
system would have to be altered to fit our game, but it wouldn't take that
much effort to come up with several weapon designs that are not only
customizable, but fun to play with. Having this kind of game element would
bring a lot of personal attachment with it, making the player more immersed.
The more fun the better, eh?
\end{subsubsection}
\end{subsection}
\end{section}

\begin{section}{Group Vision - Lance Wells}
I have decided to write this section in terms of fictional prose in an 
effort to more accurately describe the scenario without sounding too
laborious in the process.

\begin{subsection}{Setting}
Nestled between the dusty bosom of Mars and the neo-genetic warlords of
Venus lies the glorious visage of Earth, a bustling, cybernetic utopia whose
chief export - human labor - drives the many corporate houses calling the
blue, green, and gray planet home.

One of the largest organizations on Earth, the Western Organization for
Resurgent Knowledge (W.O.R.K.), is your newest employer. White-washed,
colorless cubicles blanketed by flickering, fluorescent lights stretch for
miles as your sleepless coworkers dawdle about their systems.

Fortunately, escape from this bleak business purgatory is but keystrokes
away. As an Information Technology Associate (of the future), your primary
purpose in this business is to hunt down and remove any pesky viruses
infecting the systems of your associates.

Neon-blue lines shift, pulsate, and shimmer as you enter the system of a
sales associate from cubicle 28xA1x47, Martha. Gradually coming into focus,
matrices of virtual wire outline your new surroundings. Walls and corridors
begin to form and shift into a gradually expanding labyrinth. Just as you
catch your breath, a dimly glowing red object comes into view, twitching
menacingly.

``A virus,'' you mutter to yourself as your right hand gradually shifts
downwards towards the weapon holstered on your e-belt. A lime-green,
glittering pistol reaches your fingertips. Pieced together meticulously from
parts won from previous conquests, its forking barrel snaps into action. As
you pull the trigger, four ``bit-lets'' jet towards the malware, twirling
into a tightly-closed circle, a signature of the Wrought-I/Orn hammer which
you so cleverly installed before this encounter.

Numerous foes follow, each torn apart by the partnership of your ruthless
trigger finger, and the weapon you endearingly refer to as ``Lucy''. Your
conquest ends, rather heroically, to be greeted by the familiar, boisterous
woman whose computer you have rid of the virtual invaders.

Left with the satisfaction of your coworker's appreciation, you feel your
efforts echoing across the cubicle space. Friends of your new ally, Martha,
now refer to you as a ``friend'', and promotions begin to feel inevitable.
Some day, you too aspire to face the mysterious figure at the head of
W.O.R.K., known to all as ``The C.E.O''.
\end{subsection}

\begin{subsection}{Story}

As a fresh-faced employee of the futuristic conglomerate known as
``W.O.R.K.'', your goal is to locate and destroy any viruses infecting the
systems of your coworkers. You, along with any other of your cohorts,
cleanse systems through the use of virtual reality. By cleaning each system,
you gain cybernated components for your digital arsenal in addition to 
the networked affections of your coworkers.

Your own intentions, however, lie with working your way up the corporate
ladder to eventually face the shadowy head of the company, ``The C.E.O.''.
A figure masked in infamy, little is known of The C.E.O. aside from the
effectual manner in which the company is run.

As you cut away infections from the workstations of your coworkers, the
source and origin of the malodorous programming slowly reveals itself. The
very same individual that determines the fate of your corporation is the one
distributing the electronic diseases.

Determined to stop this proliferation, you commit yourself to reaching the
corporate office and challenging your supervisor. System after system, your
skill as a malware-slayer grows in addition to the supply of friends that
begin to network and rally behind you.

Upon reaching the dark corporate ``castle'' of your dreaded executive
handler, you enter to discover that the sinister force behind the computer
plague is, itself, a computer. The final encounter of your journey, you
quickly adorn your virtual equipment and venture into the mainframe of the
baneful boss.

The swirling den of your former commander bids you death as you soon learn
that there is no easy solution to defeating this monster. A quick snippet of
code crafted by your affectionate coworkers materializes into a new shape
in your hands, the ``Information Terminator''. Like a magnet to a monitor,
you have but one option, you must lay the transmitter atop the C.E.O.'s 
sole weakness, the Malevolent Automation Node. As your mission becomes
clear, lights flash across your screen, ``To defeat this otherwise
impossible enemy, you must stick I.T. to the M.A.N.''.

The moment you strike the node with your transmitter, everything ceases to
move. All lights and motion slowly fall out of existence. Like black paint
flecked along a picture, the world around you gradually fades into
nothingness. Soon, all that is left is the booming voice of the former 
Computerized Executive Officer repeating ``ERROR: SYSTEM MALFUNCTION:
RECOMMEND REBOOT''. The cheers of your coworkers draw you back to reality,
their exuberance can only mean one thing. As shapes and colors return to
your vision, you glance at the monitor which had glared at you so
threateningly before. A happy face, composed of few pixels illustrates the
screen. Your victory is finally at hand.
\end{subsection}

\begin{subsection}{Mechanics}
For our game, we have decided to include several key features:

\begin{itemize}
\item \textbf{Weapon Creation System}

Our group has envisioned the weapons for our dungeon crawler to be 
largely user-defined. By defeating dungeons, the user is able to piece 
together various scraps and parts to form entirely new weaponry with 
which to conquer a dungeon. While we have not discussed any hard-set 
components, the concept maintains that the user will be able to craft 
new items at a weapons terminal to take with them into instances.

\item \textbf{Dungeon Lobby and Dungeon Instancing}

Our imagined dungeon crawler will utilize a single ``lobby'' in which the 
player is able to interact with their fellow office employees, group 
together with other players, and create new weapons. The lobby in question 
will resemble a modern office building, complete with grimy coffee stains 
and irritating lighting.

Entering a dungeon will consist of speaking with a given office worker, 
and fulfilling their request to clean their system of a viral infestation. 
Upon entering the ``dungeon'', the player will be joined by other players, 
and will be able to begin hunting down and exterminating any rogue programs.

\item \textbf{Employee Networking}

Employees from our group vision are the entire source and lifeblood of the 
office. Rather than implementing a generic leveling system, networking 
with employees by completing specific quests and dungeons will grant the 
player increased relations with that employee and any of their subsequent 
friends. Forming a complex web, a player's coworker relations conceive a 
unorthodox progression throughout the game, granting access to heretofore 
unseen floors, areas, etc.
\end{itemize}
\end{subsection}
\end{section}

\chapter{Project Specification}

\begin{section}{Use Cases}

\begin{subsection}{Morgan Holbart}
\begin{subsubsection}{Crafting Items}
\textbf{Actors}:

Player

\textbf{Preconditions}:

\begin{enumerate}
\item The player has the necessary ingredients to make a crafted item.
\end{enumerate}

\textbf{Summary}:

The player takes ingredients for a crafted item and expends them to
recieve a different item.

\textbf{Steps}:

\begin{enumerate}
\item Player opens crafting menu.
\item Player selects crafting recipe.
\item The items for the recipe are destroyed.
\item The crafted item is placed in the player's inventory.
\end{enumerate}
\end{subsubsection}

\begin{subsubsection}{Dialog}
\textbf{Actors}: Player, NPC

\textbf{Preconditions}:

\begin{enumerate}
\item The player is currently able to interact with the target.
\item The target wishes to interact with the player.
\item The NPC has something to say.
\end{enumerate}

\textbf{Summary}:

A player tries to initiate conversation with an NPC.

\textbf{Steps}:

\begin{enumerate}
\item The player interacts with the NPC.
\item The NPC responds to the player.
\item The player can respond based on user input.
\item When the NPC runs out of things to say, the dialog ends.
\end{enumerate}
\end{subsubsection}

\begin{subsubsection}{Opening and Closing Inventory}
\textbf{Actors}:

Player

\textbf{Preconditions}:
 
\begin{enumerate}
\item Player can currently perform the action of opening/closing inventory.
\end{enumerate}

\textbf{Steps}:

\begin{enumerate}
\item The player presses the input to open/close the inventory.
\item The inventory is loaded and displayed or closed .
\end{enumerate}
\end{subsubsection}

\begin{subsubsection}{Starting Game}
\textbf{Actors}:

Player

\textbf{Preconditions}:
 
\begin{enumerate}
\item Player is in the menu.
\item Player is selecting the start game button.
\end{enumerate}

\textbf{Steps}:

\begin{enumerate}
\item The player presses the input to start the game.
\item The game loads the user save data.
\item The game loads the lobby level with regard to the user save data.
\end{enumerate}
\end{subsubsection}

\begin{subsubsection}{Reload Weapon}
\textbf{Actors}:

Player

\textbf{Preconditions}:
 
\begin{enumerate}
\item Player has a weapon equipped that can reload.
\item Player has ammo to reload with.
\end{enumerate}

\textbf{Steps}:

\begin{enumerate}
\item Press the reload button.
\item Subtract ammo from the stored ammo.
\item Add the subtracted ammo to the current ammo.
\end{enumerate}
\end{subsubsection}

\begin{subsubsection}{Start Dungeon}
\textbf{Actors}:

Player

\textbf{Preconditions}:

\begin{enumerate}
\item Player is the leader of a dungeon group.
\item Player is currently connected with all party members.
\item Party members have all selected the ready button.
\item Party members all have adequate reputation to do the dungeon.
\end{enumerate}

\textbf{Steps}:

\begin{enumerate}
\item Player interacts with the computer he wishes to enter.
\item All other players are in party already, or join the party through the
computer.
\item The party leader presses the start dungeon button.
\end{enumerate}
\end{subsubsection}

\begin{subsubsection}{Attacking (with Ronnie Rodriguez)}
\textbf{Actors}:

Player

\textbf{Preconditions}:

\begin{enumerate}
\item Player has a valid weapon to use equipped.
\item Player has ammo or durability to use equipped weapon.
\item Player is able to attack (not in a current dialog/other interaction).
\end{enumerate}

\textbf{Steps}:

\begin{enumerate}
\item The player presses the input to attack.
\item Ammunition/Durability is subtracted.
\item The weapon animation/effects/etc are instantiated.
\item Amount of damage done per hit is subtracted from the enemy's total hit
points.
\end{enumerate}
\end{subsubsection}
\end{subsection}

\begin{subsection}{Ronnie Rodriguez}
\begin{subsubsection}{Opening Doors}
\textbf{Actors}:

Player

\textbf{Preconditions}:

\begin{enumerate}
\item The door is able to be opened.
\end{enumerate}

\textbf{Summary}:

Player tries to open a door to access area on the other side. 

\textbf{Steps}:

\begin{enumerate}
\item The player approaches the door.
\item Once close enough to the door, the player presses input to attempt to
enter.
\item If door is unlocked, door opens and player is granted access to the
area on the other side.
\item If door is locked, a "locked" or "Need Key" prompt is output to the
screen.
\end{enumerate}
\end{subsubsection}

\begin{subsubsection}{Killing an Enemy}
\textbf{Actors}:

Player, Enemy

\textbf{Preconditions}:

\begin{enumerate}
\item Player is able to attack enemy.
\item The enemy has a low enough armor rating to where the player's attacks
are damaging it.
\item Enemy can be killed.
\end{enumerate}

\textbf{Summary}:

Player attacks enemy with weapon until enemy's hit points are
completely depleted.

\textbf{Steps}:

\begin{enumerate}
\item Player gets close enough to enemy to attack either ranged or melee.
\item Player depletes enemy's hit points before player's hit points are
depleted by enemy.
\item Enemy ceases attacking and enemy death animation is instantiated.
\end{enumerate}
\end{subsubsection}

\begin{subsubsection}{Using Keys and Codes}
\textbf{Actors}:

Player

\textbf{Preconditions}:

\begin{enumerate}
\item A door the player wishes to go through is currently locked.
\item Player has needed key/code to unlock door in inventory.
\end{enumerate}

\textbf{Summary}:

Player uses found keys or codes to unlock locked doors.

\textbf{Steps}:

\begin{enumerate}
\item Player gets close enough to door to interact with it.
\item Player presses input to initialize unlocking procedure.
\item ``Unlock door with key/code?'' prompt is displayed on the screen along
with Yes or No choice.
\item Player chooses Yes or No option.
\item If player chooses No, prompt closes, door remains locked, and key/code
remains in inventory.
\item If player chooses Yes, ``Door Unlocked'' prompt displays to screen,
door unlocks, and key/code is removed from inventory.
\item Player initiates Open Door procedure (see above).
\end{enumerate}
\end{subsubsection}

\begin{subsubsection}{Taking Damage}
\textbf{Actors}:

Player

\textbf{Preconditions}:

\begin{enumerate}
\item Player is being attacked by enemy.
\item Player has a low enough armor rating to where they are able to be
damaged by current enemy.
\end{enumerate}

\textbf{Summary}:

Player is being attacked and is losing health/HP.

\textbf{Steps}:

\begin{enumerate}
\item Player is attacked by enemy.
\item Depending on strength of enemy, an amount of damage is done to the
player.
\item Amount of hit points directly related to amount of damage done by
enemy is subtracted from the player's total remaining health..
\end{enumerate}
\end{subsubsection}

\begin{subsubsection}{Moving Items in Inventory}
\textbf{Actors}:

Player

\textbf{Preconditions}:

\begin{enumerate}
\item Player has an item(s) in inventory they'd like to rearrange.
\end{enumerate}

\textbf{Summary}:

Player can manually rearrange items in inventory for better
organization and quicker use.

\textbf{Steps}:

\begin{enumerate}
\item Player presses input to open inventory.
\item Game action is paused while inventory is open.
\item Player selects ``Rearrange'' option from inventory sidebar.
\item Player presses input to select item they wish to move.
\item Item selected is highlighted.
\item Player navigates inventory and presses input to choose inventory slot
they would like to move item to.
\item If selected inventory slot is empty, item is moved to that slot,
original slot holding the item is now empty.
\item If selected inventory slot is occupied, selected item will swap places
with item in occupied slot. 
\end{enumerate}
\end{subsubsection}
\end{subsection}

\begin{subsection}{Tessa Saul}
\begin{subsubsection}{Moving through Areas}
\textbf{Actors}:

Player

\textbf{Preconditions}:

\begin{enumerate}
\item The player has the area being moved to unlocked or available to him.
\end{enumerate}

\textbf{Summary}:

The player accepts a prompt moving him to a new dungeon, floor, or
back to the lobby.

\textbf{Steps}:

\begin{enumerate}
\item The player uses an object (door/computer).
\item Player hits yes to the prompt on whether or not to use it.
\item The player is moved from the current location to the new location.
\item The new level is constructed (if a dungeon), or loaded (if the lobby)
and the player is placed in it.
\end{enumerate}
\end{subsubsection}

\begin{subsubsection}{Selling Items}
\textbf{Actors}:

Player, Shopkeeper

\textbf{Preconditions}:

\begin{enumerate}
\item The player has an item that the shopkeeper is willing to buy.
\item The shopkeeper has sufficient currency for the transaction.
\end{enumerate}

\textbf{Summary}:

Player exchanges an item for currency with a shopkeeper.

\textbf{Steps}:
 
\begin{enumerate}
\item Player approaches shopkeeper.
\item Player engages shop.
\item Player selects item from their inventory to sell.
\item Item is given to shopkeeper, currency is taken from shopkeeper and
given to player.
\end{enumerate}
\end{subsubsection}

\begin{subsubsection}{Buying Items}
\textbf{Actors}:

Player, Shopkeeper

\textbf{Preconditions}:

\begin{enumerate}
\item The shopkeeper has an item that the player is willing to buy.
\item The player has sufficient currency for the transaction.
\end{enumerate}

\textbf{Summary}:

Player exchanges an currency for item with a shopkeeper.

\textbf{Steps}:
 
\begin{enumerate}
\item Player approaches shopkeeper.
\item Player engages shop.
\item Player selects item from shopkeeper's inventory to buy.
\item Item is given to player, currency is taken from player and given to
shopkeeper.
\end{enumerate}
\end{subsubsection}

\begin{subsubsection}{Picking Up Items}
\textbf{Actors}:

Player

\textbf{Preconditions}:

\begin{enumerate}
\item The player has an empty slot big enough for the item in their
inventory.
\end{enumerate}

\textbf{Summary}:

A player's action transfers an item from the environment into their
inventory.

\textbf{Steps}:

\begin{enumerate}
\item Player approaches item.
\item Player engages item.
\item Item is removed from environment.
\item Item is placed in the player's  inventory.
\end{enumerate}
\end{subsubsection}
\end{subsection}

\begin{subsection}{Lance Wells}
\begin{subsubsection}{Die}
\textbf{Actors}:

Player

\textbf{Preconditions}:

\begin{enumerate}
\item The player is actively playing the game and has just taken
fatal damage.
\end{enumerate}

\textbf{Summary}:

The player has taken fatal damage and now proceeds to die with a
restart option.

\textbf{Steps}:

\begin{enumerate}
\item The player takes some form of fatal damage.
\item The player is then frozen on the spot, unable to interact with any
surroundings.
\item The player proceeds into a possible death animation for a few seconds.
\item The player must then choose from a pop-up menu how they would like to
proceed.
\end{enumerate}
\end{subsubsection}

\begin{subsubsection}{Leave a Dungeon}
\textbf{Actors}:

Player

\textbf{Preconditions}:

\begin{enumerate}
\item The player currently inside of a dungeon.
\end{enumerate}

\textbf{Summary}:

The player leaves a dungeon through the use of a menu.

\textbf{Steps}:

\begin{enumerate}
\item1 The player presses a button that requests to leave the current
dungeon.
\item A confirmation window appears that prompts the player if they would
like to proceed.
\item If the player chooses to leave the dungeon again, they will switch
to the main lobby.
\item If the player chooses to remain in the dungeon, the prompt window will
close.
\end{enumerate}
\end{subsubsection}

\begin{subsubsection}{Use an Item}
\textbf{Actors}:

Player

\textbf{Preconditions}:

\begin{enumerate}
\item The player currently has some usable item.
\end{enumerate}

\textbf{Summary}:

The player actively uses some usable item.

\textbf{Steps}:

\begin{enumerate}
\item The player either opens their inventory, or currently has a usable
item equipped in some fashion.
\item The player will indicate the specific item that they wish to use
through some button combination or key press.
\item The item then is removed from the player's inventory.
\item The item's effect then occurs.
\end{enumerate}
\end{subsubsection}

\begin{subsubsection}{Change Current Weapon}
\textbf{Actors}:

Player

\textbf{Preconditions}:

\begin{enumerate}
\item The player currently has two or more selectable weapons.
\end{enumerate}

\textbf{Summary}:

The player indicates a weapon to change and does so.

\textbf{Steps}:

\begin{enumerate}
\item The player either opens their inventory or currently has two or more
``equipped'' weapons.
\item The player indicates a request as to which weapon to switch to through
some key press.
\item The player's active weapon is considered to no longer be active and
potentially returned to their inventory.
\item The player then either equips the newly indicated weapon or that new
weapon is considered to be active.
\end{enumerate}
\end{subsubsection}

\begin{subsubsection}{Close the Game}
\textbf{Actors}:

Player

\textbf{Preconditions}:

\begin{enumerate}
\item The player has access to the menu that allows for the game to close.
\end{enumerate}

\textbf{Summary}:

The player closes the game via an in-game menu option.

\textbf{Steps}:

\begin{enumerate}
\item The player indicates a request to leave via some button press.
\item A pop-up window appears requesting the player to confirm that they
would like to close the game.
\item If the player should proceed, the program will (potentially) auto-save
and then close entirely.
\end{enumerate}
\end{subsubsection}

\begin{subsubsection}{Select an Item from Inventory}
\textbf{Actors}:

Player

\textbf{Preconditions}:

\begin{enumerate}
\item The player is currently at the main menu.
\end{enumerate}

\textbf{Summary}:

The player selects an option via some button press that reacts
appropriately.

\textbf{Steps}:

\begin{enumerate}
\item The player is selects some menu option with a button press.
\item The menu either then disappears and is replaced with another menu, or
closes and reacts according to the user's request.
\end{enumerate}
\end{subsubsection}

\begin{subsubsection}{Level Up}
\textbf{Actors}:

Player

\textbf{Preconditions}:

\begin{enumerate}
\item The player has all requirements to achieve an increase in level
\item The player is currently in the lobby.
\end{enumerate}

\textbf{Summary}:

The player moves (literally) up a level in progress.

\textbf{Steps}:

\begin{enumerate}
\item The player is informed through some message that they are now eligible
for an advancement in their level.
\item The player makes a request with some NPC that they would like to
advance a level.
\item The player's current reputation is now adjusted to mirror their
current level.
\item The player is physically adjusted up one level of the office building.
\end{enumerate}
\end{subsubsection}

\begin{subsubsection}{Increase Stats}
\textbf{Actors}:

Player

\textbf{Preconditions}:

\begin{enumerate}
\item The player has alterable stats that can be changed through some
action.
\end{enumerate}

\textbf{Summary}:

The player makes a request for a stat adjustment through some
method and is altered in kind.

\textbf{Steps}:

\begin{enumerate}
\item The player makes a request to change stats through an item being
equipped or unequipped or otherwise.
\item The player's request is processed and the player's stats are changed
to mirror the request.
\end{enumerate}
\end{subsubsection}
\end{subsection}

\begin{subsection}{Zachary Yama}
\begin{subsubsection}{Equip and Item}
\textbf{Actors}:

Player

\textbf{Preconditions}:
 
\begin{enumerate}
\item The player has the status to equip the item. 
\item The item is in their inventory.
\end{enumerate}

\textbf{Summary}:

A player's action transfers an item from their inventory to their
equipment.

\textbf{Steps}:

\begin{enumerate}
\item Player has the item.
\item Player selects an equip action.
\item Item is removed from inventory.
\item Item is placed in equipment.
\item Item(s) currently in equipment occupying the spaces required to eqiup
the new item are moved to inventory.
\item If the item(s) being swapped out assume more inventory space then the
player has, ignore/undue the equip action.
\end{enumerate}
\end{subsubsection}

\begin{subsubsection}{Look at Stats}
\textbf{Actors}:

Player

\textbf{Preconditions}:

\begin{enumerate}
\item The player selects the status menu icon.
\end{enumerate}

\textbf{Summary}:

Input from the player opens the status menu.

\textbf{Steps}:

\begin{enumerate}
\item Player selects stats menu icon.
\item The status window is opened.
\item Status information is read and loaded into the window.
\item Stat alteration actions are deactivated or activated depending on
status point availability.
\end{enumerate}
\end{subsubsection}

\begin{subsubsection}{Look at Map}
\textbf{Actors}:

Player

\textbf{Preconditions}:

\begin{enumerate}
\item The player toggles the map menu icon.
\end{enumerate}

\textbf{Summary}:

Input form the player opens the map menu.

\textbf{Steps}:

\begin{enumerate}
\item The player selects the map menu icon.
\item The map menu is opened.
\item Player coordinates are set as a map rendering position.
\item The surrounding level content is loaded into the map.
\item Explored content is displayed.
\end{enumerate}
\end{subsubsection}

\begin{subsubsection}{Hit a Monster}
\textbf{Actors}:

Player, Monster

\textbf{Preconditions}:

\begin{enumerate}
\item The player has attacked.
\item The player's attack hitbox hit a monster.
\end{enumerate}

\textbf{Summary}:

A player has performed the action to attack, and successfully hit
an enemy.

\textbf{Steps}:

\begin{enumerate}
\item Calculate active collisions to determine if the enemy was hit.
\item The damage delt is calculated based on player and monster attributes.
\item Damage is deducted from the monster's health. 
\item Any elemental or otherwise behavior altering elements are applied to
the monster.
\item  If the monster's health is zero, the monster's status is changed to
dead.
\end{enumerate}
\end{subsubsection}

\begin{subsubsection}{Kill a Monster}
\textbf{Actors}:

Player, Monster

\textbf{Preconditions}:

\begin{enumerate}
\item The player has reduced a monster's health to zero.
\end{enumerate}

\textbf{Summary}:

A monsters health has been reduced to zero, and is dead.

\textbf{Steps}:

\begin{enumerate}
\item The monster's hitbox is no longer active.
\item Any active AI is no longer active.
\item A death animation is played.
\item The displayed sprite is changed to represent a dead enemy.
\item The status of the enemy is set to dead, and any ressurection elements
are set to applicable.
\item The enemy is determined to be permanatly dead.
\item Any items to drop are generated.
\item The item is instanciated and dropped.
\end{enumerate}
\end{subsubsection}

\begin{subsubsection}{Level Up}
\textbf{Actors}:

Player

\textbf{Preconditions}:

\begin{enumerate}
\item The player has the required experience (Reputation?) to level up.
\end{enumerate}

\textbf{Summary}:

A player has gained enough experience to increase their level. 

\textbf{Steps}:

\begin{enumerate}
\item The experience level is compared to the amount required to level up. 
\item If the experience level is high enough, the player's level is
increased by 1.
\item Usable stats points are awarded to the stats class. 
\item The expereince bar size is recalculated.
\item The current amount of experience is reduced to zero plus any
additional experience carryover.
\item A levelup indicator is displayed.
\item Health is regenerated.
\end{enumerate}
\end{subsubsection}

\begin{subsubsection}{Increase Stats}
\textbf{Actors}:

Player

\textbf{Preconditions}:

\begin{enumerate}
\item The player has available status points to spend.
\end{enumerate}

\textbf{Summary}:

A player has unspent stats points and expends them. 

\textbf{Steps}:

\begin{enumerate}
\item The player selects a stat to increase.
\item The amount of points required to increase a stat are calculated.
\item If a stat requires more points than available, its increase action is
deactivated.
\item If there are enough points to expend, available stats points are
reduced accordingly.
\item The stat is increased. 
\item Steps 1 to 5 are repeated for every action until all stats points are
expened.
\item Any remaining points are saved.
\end{enumerate}
\end{subsubsection}
\end{subsection}
\end{section}

\end{document}
